\section{Appendix}
\subsection{Appendix A: de partiële afgeleide van J($\theta)$}
Terugkomend op de partiële afgeleide van $J(\theta)$. De afgeleide van $J(\theta)$ met respect tot $\theta_j$ is te herleiden met behulp van de differentiatieregels. Voor dit proces is het vereist om eerst de \textit{constante factor regel} toe te passen, vervolgens de machtsregel in combinatie met de kettingregel te benutten, en tot slot $h_{\theta}(x^{(i)})$ uit te schrijven om wederom de machtsregel te gebruiken. De herleiding is dan als volgt:

\[
\begin{aligned}
\frac{\partial}{\partial \theta_j}J(\theta) &= \frac{\partial}{\partial \theta_j} (\frac{1}{2m} \sum_{i=1}^{m} (( h_\theta(x^{(i)}) - y^{(i)} ) ^2), \text{ met: } 0 \le j \le n\\
& = \frac{1}{2m} \cdot \frac{\partial}{\partial \theta_j} (\sum_{i=1}^{m} ( (h_\theta(x^{(i)}) - y^{(i)} ) ^2), \text{ met: } 0 \le j \le n\\
& = \frac{2}{2m} \cdot \sum_{i=1}^{m} ( h_\theta(x^{(i)}) - y^{(i)} ) \cdot \frac{\partial}{\partial \theta_j}(\sum_{i=1}^{m} ( h_\theta(x^{(i)}) - y^{(i)} ), \text{ met: } 0 \le j \le n \\
& = \frac{2}{2m} \cdot \sum_{i=1}^{m} ( h_\theta(x^{(i)}) - y^{(i)} ) \cdot \frac{\partial}{\partial \theta_j}(\sum_{i=1}^{m} ( (\theta_0 + \theta_1x^{(i)}_1 + ... + \theta_jx^{(i)}_j) - y^{(i)} )), \text{ met: } 0 \le j \le n \\
& = \frac{1}{m} \cdot \sum_{i=1}^{m} ( h_\theta(x^{(i)}) - y^{(i)} ) \cdot x_j^{(i)}, \text{ met: } 0 \le j \le n
\end{aligned}
\]

Met dank aan Roy Voetman (https://github.com/RoyVoetman).


\subsection{Appendix B: Formalismen}


\begin{tabular}{p{1cm} || p{10cm} }
    \textbf{Wat} & \textbf{Betekenis} \\
    \hline
    $m$ & aantal observaties in de dataset \\[3pt]
    $n$ & aantal eigenschappen (\textit{properties}) per observatie \\[3pt]
    $x^{(i)}$ & observatie nummer $i$ \\[3pt]
    $x_j^{(i)}$ & de waarde van eigenschap $j$ van observatie nummer $i$ \\[3pt]
    $\theta_j$ & de waarde waarmee eigenschap $j$ door het model vermenigvuldigd wordt \\[3pt]
    $y^{(i)}$ & de werkelijke (\textit{actuele}) waarde van observatie nummer $i$ \\[3pt]
    $h_\theta^{(i)}$ & de door het model \textit{voorspelde uitkomst} op basis van de waarden van de eigenschappen van observatie nummer $i$ \\[3pt]
    $\hat{y}^{(i)}$ & hetzelfde als $h_\theta^{(i)}$ \\[3pt]
\end{tabular}
